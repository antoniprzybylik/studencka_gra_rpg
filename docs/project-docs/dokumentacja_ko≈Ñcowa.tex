\documentclass[12pt, titlepage]{article}
\usepackage[polish]{babel}
\usepackage[utf8]{inputenc}
\usepackage[T1]{fontenc}
\usepackage{nunito}
\usepackage[margin=1in]{geometry}
\usepackage{ragged2e}
\usepackage{graphicx}
\usepackage{grffile}
\usepackage{longtable}
\usepackage{wrapfig}
\usepackage{rotating}
\usepackage[normalem]{ulem}
\usepackage{amsmath}
\usepackage{textcomp}
\usepackage{amssymb}
\usepackage{capt-of}
\usepackage{hyperref}
\usepackage{listings}
\usepackage{xcolor}

\author{Antoni Przybylik \\ PROI 23L \\ GRUPA 103 \and
	Zoja Hordyńska \\ PROI 23L \\ GRUPA 103}
\date{\today}
\title{Studenckie RPG - dokumentacja końcowa projektu}

\setcounter{secnumdepth}{0}
\begin{document}
\maketitle
\justifying

\definecolor{codegreen}{rgb}{0,0.6,0}
\definecolor{codegray}{rgb}{0.5,0.5,0.5}
\definecolor{codepurple}{rgb}{0.58,0,0.82}
\definecolor{backcolour}{rgb}{0.95,0.95,0.92}

\lstdefinestyle{listingstyle}{
	backgroundcolor=\color{backcolour},
	commentstyle=\color{codegreen},
	keywordstyle=\color{magenta},
	numberstyle=\tiny\color{codegray},
	stringstyle=\color{codepurple},
	basicstyle=\ttfamily\footnotesize,
	breakatwhitespace=false,
	breaklines=true,
	captionpos=b,
	keepspaces=true,
	numbers=left,
	numbersep=5pt,
	showspaces=false,
	showstringspaces=false,
	showtabs=false,
	tabsize=2
}

\hypersetup{
	colorlinks=true,
	linkcolor=blue,
}

\lstset{style=listingstyle}

\section{Opis gry}
Studencka gra RPG to wyzwanie dla
miłośników przygód, którzy chcą zmierzyć
się z największą zmorą inżynierów - Politechniką.
Czy uda Ci się przeżyć te studia?
Przejdź się korytarzami uczelni wśród
gąszczu książek, pustych biurek i
płaczących nad kodami studentów, na ścianach
widząc istny koszmar programisty – wypisany
czerwonymi literami ERROR. Niech nie zgasi
Twego zapału oschłość Pani z dziekanatu,
dbałość o szczegóły prowadzącego czy
bezwzględność, wobec Twego zdrowia
sychicznego, sesji. Legenda miejsca
głosi, iż nikt jeszcze nie przeżył
Politechniki bez uszczerbku na zdrowiu.
Czy będziesz pierwszym, który dokona
niemożliwego? 

\section{Instrukcja obsługi}
Po uruchomieniu programu, na ekranie ukaże
się menu startowe. Aby wybrać jedną z
dostępnych opcji należy użyć strzałek
(górnej lub dolnej), a następnie wcisnąć
klawisz ENTER. "`Play"' rozpocznie grę,
"`Options"' da możliwość dostosowania
parametrów gry, "`Exit"' za to spowoduje
zamknięcie okna menu.  
\\~\\
Po wciśnięciu "`Play"' i pojawieniu się
postaci na mapie, możesz swobodnie się
po niej poruszać, używając, intuicyjnie
działających, strzałek. Po spotkaniu na
swej drodze jednego z trzech bossów,
kliknij na niego, aby rozpocząć quiz.
Wybierz odpowiedź za pomocą strzałek,
a następnie zatwierdź swój wybór klawiszem
ENTER. Wciśnij spację, aby przejść do
kolejnego pytania. Po odpowiedzeniu na
wszystkie pytania (bądź utracie wszystkich
punktów zdrowia) zamknij okno za pomocą
klawisza ESC. Nie tracąc wszystkich
punktów zdrowia, odpowiedz na wszystkie
pytania i przeżyj studia! 

W pierwotnym planie gry gracz miał
walczyć z przeciwnikami, używając do
tego przedmiotów, jednak jako zwolennicy
polubownego rozwiązywania konfliktów,
stwierdziliśmy, że dużo ciekawsze i
bardziej pasujące do klimatu studiów
będzie przeprowadzenie mini quizów
z, odpowiednimi dla każdego bossa,
pytaniami.

\newpage
\section{Architektura}
Na początku założyliśmy podział projektu
na następujące cztery moduły:
\begin{itemize}
	\item Silnik gry
	\item Ciało gry
	\item Launcher
	\item Aplikacja gry
\end{itemize}

\noindent
Ostatecznie, nie zrobiliśmy osobnej
aplikacji gry. UI i ciało gry są
zintegrowane. Zostały natomiast
wyodrębnione dodatkowo moduły
\textbf{questions\_ui} i
\textbf{questions\_core}.
Ostatecznie podział projektu na
moduły wygląda w ten sposób:
\begin{itemize}
	\item Silnik gry
	\item Ciało gry
	\item Launcher
	\item Questions UI
	\item Questions Core
\end{itemize}

\subsection{Silnik}
Silnik gry dostarcza interfejs zaprojektowany w
paradygmacie obiektowym. Jest on oparty o model
"`Sprite'owy"'. Użytkownik tworzy "`duszki"' (ang.
sprite) i określa ich interakcje z otoczeniem.
Następnie dodaje duszki do silnika i uruchamia
silnik. Silnik symuluje zachowanie duszków
podane przez użytkownika. Polimorfizm w języku C++
pozwala na
nadpisywanie w klasach dziedziczących po klasie
Sprite (duszek) funkcji wirtualnych i dzięki temu
można w duszkach umieszczać kod który zostanie
wykonany na odpowiednich zdarzeniach takich jak
zderzenie dwóch duszków lub kodu, który jest
wykonywany w każdym tyknięciu zegara.
\\~\\
Silnik został zaimplementowany tak jak założono
w dokumentacji wstępnej.

\subsection{Ciało Gry}
Ciało gry to kod implementujący właściwą
rozgrywkę. Znajdują się w nim klasy takie
jak \textbf{PlayerSprite}, \textbf{TileSprite},
\textbf{BossSprite}, \textbf{LabelSprite}
które odpowiadają kolejno
za duszka gracza, duszki bloków,
duszki przeciwników, duszki będące napisami
wyświetlanymi na ekranie.
\\~\\
Poza nimi znajduje się tam kod odpowiedzialny
za inicjalizację gry (min. wygenerowanie mapy),
oraz klasa Game.

\subsection{Launcher}
Launcher jest programem
służącym do uruchomienia gry.
Z jego pomocą rozpoczyna się
rozgrywkę, zapisuje stan gry
do pliku, a także wczytuje stan 
wcześniej zapisanej gry.
\\~\\
Ostatecznie przybrał on formę
menu gry.

\subsection{Questions UI}
Aplikacja "`Questions UI"'
jest odpowiedzialna za
wyświetlanie zadań znajdujących
się w grze.
Wyświetla się tam treść pytań
wraz z dostępnymi odpowiedziami.
Zła odpowiedź odbiera
życie graczowi, a odpowiedzenie
na wszystkie pytania przeciwnika
sprawia, że znika z mapy.
\\~\\
Ten moduł jest odpowiedzialny
tylko za wyświetlanie pytań.
Klasy przechowujące pytania znajdują
się w Questions Core.

\subsection{Questions Core}
Klasy odpowiedzialne za pytania.
Jest możliwość ich serializacji,
zapisu i wczytania z pliku.
W tym module znajdują się między
innymi następujące klasy:
\\~\\
\textbf{Klasa Player} przechowuje
liczbę punktów zdrowia gracza.
\\~\\
\textbf{Klasa Enemy} jest klasą,
opisującą wrogów. Każdy wróg posiada 
wektor indywidualnych pytań oraz
punktów zdrowia, które zabierają
graczowi za niepoprawną
odpowiedź. Metody klasy
pozwalają na zarządzanie zestawem
pytań, wyświetlaniem ich treści
oraz odpowiedzi, a także 
modyfikację parametrów gracza.
\\~\\
\textbf{Klasa Question} przechowuje
informacje dotyczące pytań - ich
treść oraz dostępne odpowiedzi.

\section{Wymagania}
Gra wymaga zainstalowanej biblioteki
SFML i jsoncpp. Działa na Linuksie. 

\section{Biblioteki i Narzędzia}
Projekt w całości został zrealizowany
w języku C++, w standardzie C++20.
Nie używaliśmy żadnych rozszerzeń
kompilatora, ale skorzystaliśmy
z jednej niestandardowej biblioteki
\textbf{unistd.h} ze standardu POSIX,
który jest implementowany przez
wszystkie powszechnie używane systemy
operacyjne (Linux, MacOS, Windows).
\\~\\
Skrypty pozwalające na zbudowanie
projektu mają są kompatybilne z
programem make i do zbudowania
projektu ma jest używany toolchain
GCC. Wszystkie moduły można zbudować
uruchamiając Makefile w katalogu game/.
Ten Makefile rekursywnie wywoła
pliki Makefile wszystkich modułów.
\\~\\
Projekt został zrealizowany przy
użyciu biblioteki SFML, biblioteki
standardowej języka C++ (libstdc++),
biblioteki standardowej języka C (libc)
i biblioteki matematycznej (libm).
Do tego skorzystaliśmy jeszcze z
biblioteki jsoncpp do serializacji
obiektów.

\section{Podział Pracy}
\textbf{Antoni}
\begin{itemize}
	\item Silnik
	\item Ciało Gry
	\item Wybrane części
	      wszystkich innych
	      modułów
\end{itemize}
 
\noindent
\textbf{Zoja}
\begin{itemize}
	\item Launcher (Menu)
	\item Aplikacja do wyświetlania pytań
	\item Klasy przechowywujące stan gry
	\item Tekstury i obrazki
\end{itemize}

\end{document}

