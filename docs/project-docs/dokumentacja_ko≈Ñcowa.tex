\documentclass[12pt, titlepage]{article}
\usepackage[polish]{babel}
\usepackage[utf8]{inputenc}
\usepackage[T1]{fontenc}
\usepackage{nunito}
\usepackage[margin=1in]{geometry}
\usepackage{ragged2e}
\usepackage{graphicx}
\usepackage{grffile}
\usepackage{longtable}
\usepackage{wrapfig}
\usepackage{rotating}
\usepackage[normalem]{ulem}
\usepackage{amsmath}
\usepackage{textcomp}
\usepackage{amssymb}
\usepackage{capt-of}
\usepackage{hyperref}
\usepackage{listings}
\usepackage{xcolor}

\author{Antoni Przybylik \\ PROI 23L \\ GRUPA 103 \and
	Zoja Hordyńska \\ PROI 23L \\ GRUPA 103}
\date{\today}
\title{Studenckie RPG - dokumentacja końcowa projektu}

\setcounter{secnumdepth}{0}
\begin{document}
\maketitle
\justifying

\definecolor{codegreen}{rgb}{0,0.6,0}
\definecolor{codegray}{rgb}{0.5,0.5,0.5}
\definecolor{codepurple}{rgb}{0.58,0,0.82}
\definecolor{backcolour}{rgb}{0.95,0.95,0.92}

\lstdefinestyle{listingstyle}{
	backgroundcolor=\color{backcolour},
	commentstyle=\color{codegreen},
	keywordstyle=\color{magenta},
	numberstyle=\tiny\color{codegray},
	stringstyle=\color{codepurple},
	basicstyle=\ttfamily\footnotesize,
	breakatwhitespace=false,
	breaklines=true,
	captionpos=b,
	keepspaces=true,
	numbers=left,
	numbersep=5pt,
	showspaces=false,
	showstringspaces=false,
	showtabs=false,
	tabsize=2
}

\hypersetup{
	colorlinks=true,
	linkcolor=blue,
}

\lstset{style=listingstyle}

\section{Opis gry}
Studencka gra RPG to wyzwanie dla miłośników przygód, którzy chcą zmierzyć się z największą zmorą inżynierów - Politechniką. Czy uda Ci się przeżyć te studia? Przejdź się korytarzami uczelni wśród gąszczu książek, pustych biurek i płaczących nad kodami studentów, na ścianach widząc istny koszmar programisty – wypisany czerwonymi literami ERROR. Niech nie zgasi Twego zapału oschłość Pani z dziekanatu, dbałość o szczegóły prowadzącego czy bezwzględność, wobec Twego zdrowia psychicznego, sesji. Legenda miejsca głosi, iż nikt jeszcze nie przeżył Politechniki bez uszczerbku na zdrowiu. Czy będziesz pierwszym, który dokona niemożliwego? 

\section{Instrukcja obsługi}
Po uruchomieniu programu, na ekranie ukaże się menu startowe. Aby wybrać jedną z dostępnych opcji należy użyć strzałek (górnej lub dolnej), a następnie wcisnąć klawisz ENTER. “Play” rozpocznie grę, “Options” da możliwość zapisania stanu gry, a także wznowienie wcześniej zapisanej gry, “Exit” za to spowoduje zamknięcie okna menu.  
\\~\\
Po wciśnięciu “Play” i pojawieniu się postaci na mapie, możesz swobodnie się po niej poruszać, używając, intuicyjnie działających, strzałek. Po spotkaniu na swej drodze jednego z trzech bossów, kliknij na niego, aby rozpocząć quiz. Wybierz odpowiedź za pomocą strzałek, a następnie zatwierdź swój wybór klawiszem ENTER. Wciśnij spację, aby przejść do kolejnego pytania. Po odpowiedzeniu na wszystkie pytania (bądź utracie wszystkich punktów zdrowia) zamknij okno za pomocą klawisza ESC. Nie tracąc wszystkich punktów zdrowia, odpowiedz na wszystkie pytania i przeżyj studia! 

\section{Zmiany względem wcześniejszych założeń}
<TODO>



\section{Architektura}
Projekt jest podzielony na cztery moduły:
\begin{itemize}
	\item Silnik gry
	\item Ciało gry
	\item Launcher
        \item Launcher pytań
	\item Aplikacja gry
\end{itemize}

\subsection{Silnik}
Silnik gry dostarcza interfejs zaprojektowany w
paradygmacie obiektowym. Jest on oparty o model
"`Sprite'owy"'. Użytkownik tworzy "`duszki"' (ang.
sprite) i określa ich interakcje z otoczeniem.
Następnie dodaje duszki do silnika i uruchamia
silnik. Silnik symuluje zachowanie duszków
podane przez użytkownika. Polimorfizm w języku C++
pozwala na
nadpisywanie w klasach dziedziczących po klasie
Sprite (duszek) funkcji wirtualnych i dzięki temu
można w duszkach umieszczać kod który zostanie
wykonany na odpowiednich zdarzeniach takich jak
zderzenie dwóch duszków lub kodu, który jest
wykonywany w każdym tyknięciu zegara.

\subsection{Ciało Gry}
Ciało gry to kod implementujący właściwą
rozgrywkę.
\\~\\
\textbf{Klasa Player} przechowuje
liczbę punktów zdrowia gracza.
\\~\\
\textbf{Klasa Enemy} jest klasą, opisującą wrogów. Każdy wróg posiada 
wektor indywidualnych pytań oraz punktów zdrowia, które zabierają graczowi za niepoprawną
odpowiedź. Metody klasy
pozwalają na zarządzanie zestawem pytań, wyświetlaniem ich treści oraz odpowiedzi, a także 
modyfikację parametrów gracza.
\\~\\
\textbf{Klasa Question} przechowują informacje dotyczące pytań - ich treść oraz dostępne odpowiedzi.


\subsection{Launcher}
Launcher jest programem służącym do uruchomienia gry.
Z jego pomocą rozpoczyna się rozgrywkę, zapisuje stan gry do pliku, a także wczytuje stan 
wcześniej zapisanej gry.

\subsection{Launcher pytań}
Launcher jest programem odpowiednio zarządzającym, dostępnym w grze, quizem.
Wyświetla się tam treść pytań wraz z dostępnymi odpowiedziami. Zła odpowiedź odbiera
życie graczowi, a odpowiedzenie na wszystkie pytania przeciwnika sprawia, że znika z mapy.

\subsection{Aplikacja Gry}
Aplikacja gry jest programem który służy
do grania w grę. W oknie aplikacji gry jest
renderowana symulacja gry. Aplikacja gry
powinna dodatkowo pozwalać na zatrzymanie
gry, zmianę ustawień i wyjście z gry.

\section{Wymagania}
Gra wymaga zainstalowanej biblioteki
SFML. Założone jest działanie gry
na systemach operacyjnych Windows i
Linux. Opcjonalnie gra mogłaby wspierać
system MacOS.

\section{Biblioteki i Narzędzia}
Projekt w całości ma zostać zrealizowany
w języku C++, w standardzie C++20.
Dopuszczalne jest używanie rozszerzeń
GCC.
\\~\\
Skrypty pozwalające na zbudowanie
projektu mają być kompatybilne z
programem make i do zbudowania
projektu ma zostać użyty toolchain
GCC. Do kompilacji gry na Windowsa
zostanie użyty MinGW.
\\~\\
Projekt ma zostać zrealizowany przy
użyciu biblioteki SFML, biblioteki
standardowej języka C++ (libstdc++),
biblioteki standardowej języka C (libc)
i biblioteki matematycznej (libm).
\\~\\
Dodatkowe aplikacje takie jak
launcher mogą zostać zrealizowane
przy użyciu innych bibliotek np.
QT.
\\~\\
W projekcie nie jest wymagane ścisłe
trzymanie się kanonicznej wersji tych
bibliotek. Dopuszczalne jest użycie
rozszerzeń GLIBC.

\section{Podział Pracy}
\textbf{Antoni}
\begin{itemize}
	\item Silnik
	\item Wybrane części gry
\end{itemize}
 
\noindent
\textbf{Zoja}
\begin{itemize}
	\item Menu
	\item Interfejs pytań
	\item Tekstury i obrazki
	\item Ciało gry; Klasy: Player, Enemy, Question
\end{itemize}

\end{document}

