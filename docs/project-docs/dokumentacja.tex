\documentclass[12pt, titlepage]{article}
\usepackage[polish]{babel}
\usepackage[utf8]{inputenc}
\usepackage[T1]{fontenc}
\usepackage{nunito}
\usepackage[margin=1in]{geometry}
\usepackage{ragged2e}
\usepackage{graphicx}
\usepackage{grffile}
\usepackage{longtable}
\usepackage{wrapfig}
\usepackage{rotating}
\usepackage[normalem]{ulem}
\usepackage{amsmath}
\usepackage{textcomp}
\usepackage{amssymb}
\usepackage{capt-of}
\usepackage{hyperref}
\usepackage{listings}
\usepackage{xcolor}

\author{Antoni Przybylik \\ PROI 23L \\ GRUPA 103 \and
	Zoja Hordyńska \\ PROI 23L \\ GRUPA 103}
\date{\today}
\title{<TODO: Nazwa Gry> - Dokumentacja Projektu}

\setcounter{secnumdepth}{0}
\begin{document}
\maketitle
\justifying

\definecolor{codegreen}{rgb}{0,0.6,0}
\definecolor{codegray}{rgb}{0.5,0.5,0.5}
\definecolor{codepurple}{rgb}{0.58,0,0.82}
\definecolor{backcolour}{rgb}{0.95,0.95,0.92}

\lstdefinestyle{listingstyle}{
	backgroundcolor=\color{backcolour},
	commentstyle=\color{codegreen},
	keywordstyle=\color{magenta},
	numberstyle=\tiny\color{codegray},
	stringstyle=\color{codepurple},
	basicstyle=\ttfamily\footnotesize,
	breakatwhitespace=false,
	breaklines=true,
	captionpos=b,
	keepspaces=true,
	numbers=left,
	numbersep=5pt,
	showspaces=false,
	showstringspaces=false,
	showtabs=false,
	tabsize=2
}

\lstset{style=listingstyle}

\section{Cel i opis projektu}
Przedmiotem projektu jest "`<TODO: Nazwa Gry>"' - studencka
gra RPG inspirowana przez Diablo. Gracz (student) przechodzi
przez mapę Politechniki zbierając przedmioty i walcząc z potworami
godnymi studiów. Każdy gracz posiada pewne umiejętności,
zdrowie i manę. Celem gry jest zabicie wszystkich bosów
i tym samym przeżycie studiów.
\\~\\
Projekt ma zostać zrealizowany w języku C++ z użyciem
jego zaawansowanych mechanizmów i ze szczególnym uwzględnieniem
paradygmatu obiektowego. Planowane jest stworzenie
dedykowanego silnika gry zaimplementowanego przy użyciu
biblioteki SFML, który będzie warstwą abstrakcji między
wysokopoziomowym kodem właściwej gry operującym na "`duszkach"',
a interfejsem biblioteki SFML pozwalającym na proste operacje
takie jak wyświetlanie obrazów.

\section{Architektura}
Projekt jest podzielony na cztery moduły:
\begin{itemize}
	\item Silnik gry
	\item Ciało gry
	\item Launcher
	\item Aplikacja gry
\end{itemize}

\noindent
Silnik gry dostarcza interfejs zaprojektowany w
paradygmacie obiektowym. Jest on oparty o model
"`Sprite'owy"'. Użytkownik tworzy "`duszki"' (ang.
sprite) i określa ich interakcje z otoczeniem.
Następnie dodaje duszki do silnika i uruchamia
silnik. Silnik symuluje zachowanie duszków
podane przez użytkownika. Programowanie obiektowe
pozwala na przeładowanie klasy Sprite (duszek) i
umieszczenie w nim kodu który zostanie wykonany na
odpowiednich zdarzeniach takich jak zderzenie dwóch
duszków lub kodu który jest wykonywany w każdym
tyknięciu zegara.

\begin{lstlisting}[language=C++, caption=Interfejs silnika - kod poglądowy]
        /* Utworzenie duszka. */
        skin = new SpriteSkin(*tiles, 3, 500, 1);
        sprite = new Sprite(skin,
                            Rect(0, 0, win_x, win_y),
                            BA_BOUNCE);
        sprite->set_position(Rect(100, 100, 75, 75));
        sprite->set_velocity(Vector(120, 250));

        /* Silnik. */
        engine = new Engine();
        engine->bind_window(window);
        engine->add_sprite(sprite);

        engine->exec();
\end{lstlisting}

\noindent
Ciało gry to kod implementujący właściwą
rozgrywkę. <TODO: Napisz coś o ciele gry.>
\\~\\
Launcher jest programem służącym do
uruchomienia gry. Można w nim modyfikować
ustawienia rozgrywki i powinna być możliwość
wczytania stanu gry z pliku. <TODO: Napisz coś
o launcherze.>
\\~\\
Aplikacja gry jest programem który służy
do grania w grę. W oknie aplikacji gry jest
renderowana symulacja gry. Aplikacja gry
powinna dodatkowo pozwalać na zatrzymanie
gry, zmianę ustawień i wyjście z gry. <TODO:
Napisz coś o aplikacji gry.>

\section{Wymagania}
Gra wymaga zainstalowanej biblioteki
SFML. Założone jest działanie gry
na systemach operacyjnych Windows i
Linux. Opcjonalnie gra mogłaby wspierać
system MacOS. Pewnie Makefile dla Linuksa
będzie działał na MacOS, ale nie mam tego
systemu i nie zobowiązuję się że gra będzie
na nim prawidłowo działać i kompilować
bez ostrzeżeń.
\\~\\
<TODO: Napisać więcej.>

\section{Instrukcja użycia}
<TODO>

\end{document}
